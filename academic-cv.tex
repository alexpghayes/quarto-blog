\documentclass[11pt]{article}
\usepackage[letterpaper,margin=0.5in]{geometry}
\usepackage{hyperref}
\usepackage{enumitem}
\usepackage{fontawesome}
\usepackage{titlesec}
\usepackage{xcolor}
\usepackage{soul}
\usepackage{fancyhdr}
\usepackage{parskip} % For paragraph spacing control

\usepackage[sfdefault]{carlito}

% Define Dark Red 2 color (RGB: 139,0,0)
\definecolor{darkred2}{RGB}{139,0,0}

% Remove paragraph indentation
\setlength{\parindent}{0pt}

\renewcommand{\labelitemi}{--}

% Custom formatting with dark red headers
\titleformat{\section}{\Large\bfseries\color{darkred2}}{\thesection}{1em}{}[\color{gray}\titlerule]
\titlespacing*{\section}{0pt}{14pt}{10pt}

% Remove page numbers
% \pagenumbering{gobble}

\pagestyle{fancy}
\setlength{\footskip}{-8pt}

\fancypagestyle{default}{
  \fancyhf{}% No header/footer
  \renewcommand{\headrulewidth}{0pt}% No header rule
}
\fancypagestyle{last-page}{
  \fancyhf{}% No header/footer
  \renewcommand{\headrulewidth}{0pt}
  \fancyfoot[R]{\small Last updated on 2024-10-25}
}

\pagestyle{default}% Default page style
\AtEndDocument{\thispagestyle{last-page}}% Page style at \end{document}

\setlist[itemize]{
    itemsep=0.35em,     % Reduces space between items
    parsep=0em,        % Removes space between paragraphs within items
    topsep=0em       % Reduces space before and after list
}

\setlist[enumerate]{
    itemsep=0.35em,     % Reduces space between items
    parsep=0em,        % Removes space between paragraphs within items
    topsep=0em       % Reduces space before and after list
}

% Hyperref setup for black, underlined links
\hypersetup{
    colorlinks=true,
    linkcolor=black,
    urlcolor=black,
    pdfborderstyle={/S/U/W 1} % Adds PDF underlining
}

% Custom command for underlined links
\newcommand{\underlinedlink}[2]{\href{#1}{\ul{#2}}}

\begin{document}

\begin{center}
    {\huge\textbf{Alex Hayes}}

    \vspace{0.1cm}
    alexpghayes@gmail.com $\cdot$ 775 338 8842 $\cdot$ Madison, WI, USA $\cdot$ \underlinedlink{http://www.alexpghayes.com}{www.alexpghayes.com} $\cdot$ \underlinedlink{https://github.com}{github}
\end{center}

\section*{PROFESSIONAL EXPERIENCE}

 {\large \textbf{University of Wisconsin-Madison}} \hfill \textbf{August 2018 -- Present}\\
PhD Candidate, Department of Statistics
\begin{itemize}
    \item Developed statistical methods to cluster networks with missing data, to perform regression on networks, and to construct, interpret, and regularize network embeddings.
    \item Developed causal inference methods to estimate mediation and spillover effects in social networks, and to determine when product changes have harmful side-effects on behaviors that are difficult to measure. Used causal machine learning to improve precision of estimates while reducing computational requirements by a factor of 5000.
    \item Implemented research methods in user-friendly software. Released nine open source R packages to CRAN.
    \item Resolved computational bottlenecks in matrix completion algorithms by designing and implementing sparse matrix methods in R and C++. Scaled methods by three orders of magnitude to handle networks with millions of nodes.
    \item Designed an approach to find localized clusters of Twitter users via Personalized PageRank. Managed unreliable Twitter API behavior by caching data in a Neo4J database running in Docker.
\end{itemize}

{\large \textbf{Facebook}} \hfill \textbf{Summer 2020 \& Summer 2021}\\
Research Intern, Core Data Science
\begin{itemize}
    \item Prototyped a pipeline to automatically suggest relationships between hashtags, replacing a manual labeling workflow. Prototype embedded hashtag co-occurrence network and was implemented in Python, PyTorch and SQL.
    \item Conducted experiments on hyperbolic embeddings for knowledge graphs and determined non-viability of hyperbolic methods. Advised against additional R\&D investment, potentially saving \$200k+ in compute costs.
    \item Designed a metric, based on calibration of machine learning models, to help product teams understand reliability of prevalence estimates. Metric reported daily on multiple dashboards. Implemented with sklearn, Numpy, pandas.
\end{itemize}

{\large \textbf{RStudio}} \hfill \textbf{Summer 2018}\\
Intern, tidymodels team
\begin{itemize}
    \item Re-factored thousands of lines of R code and developed a new test suite for the \underlinedlink{https://github.com/tidymodels/broom}{broom} package (600k+ downloads/month, part of the tidyverse), improving behavioral consistency and reducing maintenance burden.
    \item Shipped a major new release of the package (\underlinedlink{https://www.tidyverse.org/blog/2018/07/broom-0-5-0/}{broom 0.5.0}). Resolved 80+ open issues and coordinated 40+ pull requests from open source contributors.
\end{itemize}

{\large \textbf{Rice University}} \hfill \textbf{Fall 2017}\\
Undergraduate researcher with Genevera Allen

    {\large \textbf{Fred Hutchinson Cancer Research Center}} \hfill \textbf{Summer 2017}\\
Undergraduate researcher with Elizabeth Brown

    {\large \textbf{Houston Parks and Recreation Department}} \hfill \textbf{Spring 2016}\\
Undergraduate researcher

\section*{EDUCATION}

\textbf{University of Wisconsin-Madison} \hfill 2018 -- 2024\\
Ph.D. Statistics, advised by \emph{Keith Levin} and \emph{Karl Rohe}

\textbf{Rice University} \hfill 2014 -- 2018\\
B.A. Statistics, with \emph{Distinction in Research and Creative Work}


\section*{WORKING PAPERS}

\begin{enumerate}
    \item \textbf{Alex Hayes} and Kevin Levin. \underlinedlink{https://arxiv.org/abs/2410.10772}{Peer effects in the linear-in-means model may be inestimable even when identified.} \emph{arXiv}. 2024.
    \item \textbf{Alex Hayes}, Mark M. Fredrickson, and Keith Levin. \underlinedlink{http://arxiv.org/abs/2212.12041}{Estimating network-mediated causal effects via principal components network regression.} \emph{Journal of Machine Learning Research (accepted with minor revisions)}. 2024+.
\end{enumerate}

\section*{PUBLICATIONS}


\begin{enumerate}
    \item \textbf{Alex Hayes} and Karl Rohe. \underlinedlink{http://www.tandfonline.com/10.1080/10618600.2024.2394464}{Co-factor analysis of citation networks}. \emph{Journal of Computational and Graphical Statistics}. 2024.
    \item Hadley Wickham, Mara Averick, Jennifer Bryan, Winston Chang, Lucy D'Agostino McGowan, Romain François, Garrett Grolemund, \textbf{Alex Hayes}, Lionel Henry, Jim Hester, Max Kuhn, Thomas Lin Pedersen, Evan Miller, Kirill Müller, David Robinson, Dana Paige Seidel, Vitalie Spinu, Kohske Takahashi, Davis Vaughan, Claus Wilke, Kara Woo, Hiroaki Yutani. \underlinedlink{https://joss.theoj.org/papers/10.21105/joss.01686}{Welcome to the Tidyverse}. \emph{Journal of Open Source Software.} 2019.
\end{enumerate}

\section*{RESEARCH \& STATISTICAL SOFTWARE}

\begin{enumerate}
    \item \underlinedlink{https://CRAN.R-project.org/package=fastRG}{fastRG} (CRAN, Github): Sample generalized random dot product graphs linearly in edge counts. Useful to investigate properties of network models and spectral estimators. Easily control expected degree of sampled networks, and efficiently compute population eigendecompositions for large networks.

    \item \underlinedlink{https://CRAN.R-project.org/package=vsp}{vsp} (CRAN, Github): Vintage sparse PCA for semi-parametric network analysis. Estimate latent positions in random dot product graphs via spectral embeddings and interpret them via varimax rotation. Easily regularize networks to handle noise.

    \item \underlinedlink{https://rohelab.github.io/aPPR/}{aPPR} (Github): Approximate personalized pageRank. Locally clusters networks based on degree-regularized PageRank estimates. Designed specifically for large networks only available via an API.

    \item \underlinedlink{https://CRAN.R-project.org/package=gdim}{gdim} (CRAN, Github): Estimate graph dimension using cross-validated eigenvalues. Determine the number of communities in stochastic blockmodels and variants.

    \item \underlinedlink{https://CRAN.R-project.org/package=fastadi}{fastadi} (CRAN, Github): Self-tuning matrix imputation. Estimating singular subspaces of sparsely observed matrices. Includes specialized methods for upper triangular data.

    \item \underlinedlink{https://CRAN.R-project.org/package=broom}{broom} (CRAN, Github): Convert statistical objects into tidy tibbles. Part of the tidyverse. Puts hundreds of types of statistical estimates into a consistent format to make programming easier.

    \item \underlinedlink{https://CRAN.R-project.org/package=distributions3}{distributions3} (CRAN, Github): Probability distributions as S3 Objects. An object-oriented interface to probability computations, with emphasis on careful documentation, beginner friendliness and classroom applicability.
\end{enumerate}

\vspace{0.2cm}
In addition to writing code, I collaborated with ROpenSci to design software development standards for statistical software, I review scientific software for ROpenSci, the R Journal, and the Journal of Open Source Software, and I helped organize the Chicago R Unconference in 2019.

\section*{TALKS}

\begin{enumerate}
    \item Estimating network-mediated causal effects via spectral embeddings  \hfill 2024-06-17 \\ \emph{\small NetSci 2024}
    \item Asymptotic identification of peer effects in linear models \hfill 2024-04-04 \\  \emph{\small Dissertation defense}
    \item Peer effects are parametrically indistinguishable from baseline behaviors in the asymptotic limit \hfill 2023-11-27 \\ \emph{\small Statistics Graduate Student Seminar, UW-Madison}
    \item Latent contagion in low-rank networks \hfill 2023-10-11 \\  \emph{\small Levin Lab Meeting, UW-Madison}
    \item Peer diffusion over uncertain networks \hfill 2023-09-18 \\ \emph{\small IFDS Ideas Seminar, UW-Madison}
    \item Estimating network-mediated causal effects via spectral embeddings \hfill 2023-08-09 \\ \emph{\small JSM 2023}
    \item Estimating network-mediated causal effects via spectral embeddings  \hfill 2023-04-24 \\ \emph{\small IFDS Ideas Seminar, UW-Madison}
    \item Estimating network-mediated causal effects via spectral embeddings \hfill 2022-10-14 \\ \emph{\small Statistics Graduate Student Seminar, UW-Madison}
    \item Estimating indirect effects induced by homophily via spectral network regression. \hfill 2022-07-07 \\ \emph{\small Tianxi Li and Can Le Joint Lab Meeting}
    \item distributions3: From basic probability to probabilistic regression \hfill 2022-06-23 \\ \emph{\small UseR 2022}
    \item The Low Hanging Fruit of the Twitter Following Graph \hfill 2021-08-11 \\ \emph{\small JSM 2021}
    \item Solving the model representation problem with broom \hfill 2019-01-25 \\ \emph{\small rstudio::conf(2019)}
    \item Solving the model representation problem with broom \hfill 2018-11-30 \\ \emph{\small Statistics Graduate Student Seminar, UW-Madison.}
    \item Convenient data analysis with broom \hfill 2018-11-14 \\ \emph{\small RStudio Webinar Series}
    \item Solving the model representation problem with broom \hfill 2018-09-19 \\ \emph{\small Madison R User Group}
\end{enumerate}

\section*{POSTER PRESENTATIONS}

\begin{enumerate}
    \item Estimating network-mediated causal effects via spectral embeddings  \hfill 2023-08-07 \\ \emph{IFDS Annual Meeting}
    \item Estimating network-mediated causal effects via spectral embeddings \hfill 2023-05-24 \\ \emph{ACIC 2023}
    \item Using data to support real-time decision making by the Hurricane Harvey crisis management team \hfill 2017-10-10 \\ \emph{Rice Data Science Conference}
    \item An exploratory analysis of the effect of waiting room interactions on adherence in clinical trials \hfill 2017-08-10 \\ \emph{Fred Hutch Intern Poster Competition}
\end{enumerate}

\section*{TEACHING}

\textbf{Graduate Teaching Assistant}
\begin{itemize}
    \item STAT 340 Intro to Data Modeling II \hfill Fall 2022 \\ \emph{\small UW-Madison}
    \item STAT 324 Intro to Statistics for Engineers \hfill Spring 2019 \\ \emph{\small UW-Madison}
    \item STAT 324 Intro to Statistics for Engineers \hfill Fall 2018 \\ \emph{\small UW-Madison}
    \item Statistics Department Outstanding TA Award \hfill 2018-2019 \\ \emph{\small UW-Madison}
\end{itemize}

\textbf{Guest Lecturer}
\begin{itemize}
    \item Confidence intervals. STAT 340 \hfill 2022-10-25 \& 2022-10-27 \\ \emph{\small UW-Madison}
    \item Sampling with Twitter following graph with aPPR. STAT 992 \hfill 2020-10-08 \\ \emph{\small UW-Madison}
    \item Hypothesis testing. STAT 324 \hfill 2018-10-18 \\ \emph{\small UW-Madison}
\end{itemize}

\textbf{Co-instructor}
\begin{itemize}
    \item Applied Machine Learning Workshop \hfill 2019-01-15 \& 2019-01-16 \\ \emph{\small rstudio::conf(2019)}
\end{itemize}

\textbf{Undergraduate Teaching Assistant}
\begin{itemize}
    \item COMP 540 Statistical Machine Learning \hfill Spring 2018 \\ \emph{\small Rice University}
    \item COMP 330 Data Science: Tools \& Models \hfill Fall 2017 \\ \emph{\small Rice University}
\end{itemize}

\section*{MENTORING}

\begin{itemize}
    \item Nathan Kolbow (undergraduate research assistant), currently a PhD student in Biostatistics at UW-Madison
\end{itemize}

\end{document}